%%%%%%% DO NOT REMOVE COMMANDS OR PACKAGE FROM THE DOCUMENT

\documentclass[letterpaper,12pt]{article}

\usepackage{alltt}
\usepackage{amsfonts}
\usepackage{amsmath}
\allowdisplaybreaks[1]%0-4 4 is most permissive of breaking equations
\usepackage{amssymb}
\usepackage{amsthm}
\usepackage{array}
\usepackage{calc}
\usepackage{color}
\usepackage{graphicx}
\usepackage{stmaryrd}
\usepackage{supertabular}
\usepackage{url}%Wanted by the bibilography
\usepackage[all]{xy}

%%Packages for tables
\usepackage{wrapfig}
\usepackage{multirow}
\usepackage{tabu}

%Note: Throughout the preamble there are occassional references to pages out of
%books. Unless otherwise noted they are from The LaTeX Companion 2nd ed by
%Mittelbach and Goossens

%######################################
%Font Selection.
%Concrete
%\usepackage{beton}
%\usepackage{euler}
%Computer Modern Bright
%\usepackage[T1]{fontenc}
%\usepackage{cmbright}
%######################################

%######################################
%Quote environment
\newsavebox{\quauth}
\newenvironment{Quotation}[1]
	{\sbox{\quauth}{\textit{#1}}\begin{quote}}
	{\hspace*{\fill}\nolinebreak[1]\hspace*{\fill}\usebox{\quauth}\end{quote}}
%######################################

%######################################
%Theorem setup, needs package amsthm
%Theorems
\theoremstyle{plain}
\newtheorem{thm}{Theorem}[section]
\newtheorem{algo}[thm]{Algorithm}
\newtheorem{cor}[thm]{Corollary}
\newtheorem{lem}[thm]{Lemma}
\newtheorem{prop}[thm]{Proposition}


%Definitions
\theoremstyle{definition}
\newtheorem{appl}[thm]{Application}
\newtheorem{conj}[thm]{Conjecture}
\newtheorem{defn}[thm]{Definition}
\newtheorem{exmp}[thm]{Example}
\newtheorem{exer}[thm]{Exercise}

%Remarks
\theoremstyle{remark}
\newtheorem{rem}[thm]{Remark}
\newtheorem{note}[thm]{Note}
\newtheorem*{case}{Case}
\newtheorem*{claim}{Claim}


%For referencing Theorems/Defs/etc...
\providecommand{\thmref}[1]{ (Thm. \ref{#1})}
\providecommand{\defref}[1]{ (Def. \ref{#1})}
\providecommand{\exref}[1]{Example \ref{#1}}
%######################################

%######################################
%Define a few column types for convenience
%Needs package array

%These make the columns all be in math mode
%The extra %stopzone are to make vim do syntax
%highlighting correctly with this code, otherwise
%it chokes
\newcolumntype{C}{>{$}c<{$}}
%stopzone%stopzone%stopzone
\newcolumntype{L}{>{$}l<{$}}
%stopzone%stopzone%stopzone
\newcolumntype{R}{>{$}r<{$}}
%stopzone%stopzone%stopzone
%######################################

%######################################
%Various shortcuts
%Some may need package amssymb

%Shortcuts for various sets
\providecommand{\A}{\ensuremath{\mathfrak{A}}}%ideal A
\providecommand{\B}{\ensuremath{\mathcal{B}}}%basis
\providecommand{\C}{\ensuremath{\mathbb{C}}}%complex
\providecommand{\F}{\ensuremath{\mathbb{F}}}%finite field
\providecommand{\N}{\ensuremath{\mathbb{N}}}%natural
\providecommand{\RI}{\ensuremath{\mathcal{O}}}%ring of integers
\providecommand{\p}{\ensuremath{\mathfrak{P}}}%prime ideal
\providecommand{\Q}{\ensuremath{\mathbb{Q}}}%rationals
\providecommand{\R}{\ensuremath{\mathbb{R}}}%reals
\providecommand{\T}{\ensuremath{\mathcal{T}}}%topology
\providecommand{\Z}{\ensuremath{\mathbb{Z}}}%integers
\providecommand{\Zpos}{\ensuremath{\mathbb{Z}^{+}}}%positive integers


%Miscellaneous shortcuts
\providecommand{\lcm}{\ensuremath{\text{lcm}}}%least common multiple
\providecommand{\st}{\ensuremath{\backepsilon}}
\providecommand{\ud}{\ensuremath{\,\,\mathrm{d}}}%For derivatives
%######################################

%######################################
% Various operators
% some may need package amsmath

%Algebra
\providecommand{\amod}{\ensuremath{\diagup}}%Algebraic mod
\providecommand{\embed}{\ensuremath{\hookrightarrow}}%inclusion map
\providecommand{\gen}[1]{\ensuremath{\left<#1\right>}}%Group generated by #1
\providecommand{\idx}[2]{\ensuremath{\left[#1:#2\right]}}%index of #2 in #1
\providecommand{\im}{\ensuremath{\mathrm{im}}}%Image
\providecommand{\iso}{\ensuremath{\simeq}}%isomorphism
\providecommand{\normal}{\ensuremath{\vartriangleleft}}%Normal subgroup
\providecommand{\subgp}{\ensuremath{\le}}%Sub group
\providecommand{\vect}[1]{\ensuremath{\left\langle#1\right\rangle}}%Vector

%Logic
\providecommand{\land}{\ensuremath{\wedge}}
\DeclareMathSymbol\lnot{\mathbin}{symbols}{"3A}%p528
\providecommand{\lor}{\ensuremath{\vee}}
\providecommand{\lxor}{\ensuremath{\oplus}}

%Number Theory
\providecommand{\jacobi}[2]{\ensuremath{\left(\frac{#1}{#2}\right)}}
\providecommand{\legendre}[2]{\ensuremath{\left(\frac{#1}{#2}\right)}}

%Set Theory
\providecommand{\intersect}{\ensuremath{\bigcap}}
\providecommand{\less}{\ensuremath{\diagdown}}
\providecommand{\union}{\ensuremath{\bigcup}}

%Miscellaneous
\providecommand{\abs}[1]{\ensuremath{\left\lvert#1\right\rvert}}
\providecommand{\card}[1]{\ensuremath{\left\lvert#1\right\rvert}}
\providecommand{\ceiling}[1]{\ensuremath{\left\lceil#1\right\rceil}}
\providecommand{\define}{\ensuremath{\stackrel{\text{\tiny def}}{=}}}
\providecommand{\floor}[1]{\ensuremath{\left\lfloor#1\right\rfloor}}
\providecommand{\norm}[1]{\ensuremath{\left\lVert#1\right\rVert}}%Analysis norm
\providecommand{\order}[1]{\ensuremath{\left\lvert#1\right\rvert}}

%Asymptotic
\providecommand{\Oh}{\ensuremath{\mathcal{O}}}
\providecommand{\oh}{\ensuremath{o}}
%######################################
%######################################
%Page layout
% If using hyperef, load it before this block
\usepackage[paper=letterpaper,tmargin=1in,bmargin=42pt,lmargin=1in,rmargin=1in,headheight=30pt,headsep=30pt,footskip=20pt]{geometry}
% Note 1in = 72pt, therefore bmargin=42pt is 1in - 30pt
\usepackage{ifpdf}
\ifpdf
  \geometry{driver=pdftex}
\else
  \geometry{driver=dvips}
\fi

%%%%%%%%%%%%
%%%%%%%%%%%%
%%%%%%%%%%%%
%%%%%%%%%%%% Here you enter the running title
%
%%%%%%%%%%%%
\usepackage{fancyhdr}
\lhead{COMP 2300}
%It is recommended to add your name
%%%%%%%%%%%%
\chead{Homework 1 Student \#1}
\rhead{Fall 2023}
\lfoot{}
\cfoot{}
\rfoot{\thepage}
\renewcommand{\headrulewidth}{0pt}
\renewcommand{\footrulewidth}{0pt}
%######################################
%######################################
%######################################
%######################################
%######################################

\begin{document}
\pagestyle{fancy}
%######################################
%######################################
%######################################
%######################################
%HOMEWORK START


\begin{enumerate}
\item (30 points)
  Construct truth tables for the following compound propositions.
  \begin{enumerate}
    \item $\lnot p \to q$
  \item $\bigl((p \to r) \land r \bigr) \lor \lnot (\lnot p \lor r)$
  \item $\bigl((p \leftrightarrow r) \lxor q\bigr) \land (\lnot p \lor q \lor \lnot r)$
  \item $\lnot\Bigl( \bigl((\lnot p \lor r \lor q) \land (p \to r)\bigr) \lxor r\Bigr)$
  \end{enumerate}
  Are any of the propositions equivalent? If so, which ones and why?

 
(Solution) %The code for the tables is provided here. Please enter the correct truth Values into the table. Feel free to change the tables.
\begin{enumerate}
\item $\lnot p \to q$

\begin{center}
\begin{tabular}{ | c | c | c |} 
  \hline
 $p$ $q$ & $ \lnot p$ & $\lnot p \to r$ \\
 \hline
  T T & ? & ? \\ 
  T F & ? & ? \\ 
  F T & ? & ? \\
  F F & ? & ? \\
  \hline
\end{tabular}
\end{center}



\item Truth Table for $\bigl((p \to r) \land r \bigr) \lor \lnot (\lnot p \lor r)$
%
\begin{center}
\begin{tabular}{ | c | c | c | c | c | c | c |} 
  \hline
 $p$ $q$ $r$ & $p \to r$ & $(p \to r) \land r$ & $\lnot p$ & $\lnot p \lor r$ & $\lnot (\lnot p \lor r)$ & $\bigl((p \to r) \land r \bigr) \lor \lnot (\lnot p \lor r)$\\
  \hline
  T T T & ? & ? & ? & ? & ? & ?\\ 
  T T F & ? & ? & ? & ? & ? & ?\\ 
  T F T & ? & ? & ? & ? & ? & ?\\
  T F F & ? & ? & ? & ? & ? & ?\\
  F T T & ? & ? & ? & ? & ? & ?\\
  F T F & ? & ? & ? & ? & ? & ?\\
  F F T & ? & ? & ? & ? & ? & ?\\
  F F F & ? & ? & ? & ? & ? & ?\\
  \hline
\end{tabular}
\end{center}

\item Truth Table for $\bigl((p \leftrightarrow r) \lxor q\bigr) \land (\lnot p \lor q \lor \lnot r)$

\begin{center}
\begin{tabular}{ | c | c | c | c | c | c | c |} 
  \hline
 $p$ $q$ $r$ & $p \leftrightarrow r$ & $(p \leftrightarrow r) \lxor q$ & $\lnot p$ & $\lnot r$ & $\lnot p \lor q \lor \lnot r$ & $\bigl((p \leftrightarrow r) \lxor q\bigr) \land (\lnot p \lor q \lor \lnot r)$\\
  \hline
  T T T & ? & ? & ? & ? & ? & ?\\ 
  T T F & ? & ? & ? & ? & ? & ?\\ 
  T F T & ? & ? & ? & ? & ? & ?\\
  T F F & ? & ? & ? & ? & ? & ?\\
  F T T & ? & ? & ? & ? & ? & ?\\
  F T F & ? & ? & ? & ? & ? & ?\\
  F F T & ? & ? & ? & ? & ? & ?\\
  F F F & ? & ? & ? & ? & ? & ?\\
  \hline
\end{tabular}
\end{center}

\item Truth Table for $\lnot\Bigl( \bigl((\lnot p \lor r \lor q) \land (p \to r)\bigr) \lxor r\Bigr)$

\begin{center}
\begin{tabular}{ | c | c | c | c | m{2.1cm} | m{2.6cm} | m{2.8cm} |} 
  \hline
 $p$ $q$ $r$ & $\lnot p$ & $\lnot p \lor r \lor q$ & $p \to r$ & $(\lnot p \lor r \lor q) \land (p \to r)$ & $((\lnot p \lor r \lor q) \land (p \to r)\bigr) \lxor r$ & $\lnot\Bigl( \bigl((\lnot p \lor r \lor q) \land (p \to r)\bigr) \lxor r\Bigr)$\\
  \hline
  T T T & ? & ? & ? & ? & ? & ?\\ 
  T T F & ? & ? & ? & ? & ? & ?\\ 
  T F T & ? & ? & ? & ? & ? & ?\\
  T F F & ? & ? & ? & ? & ? & ?\\
  F T T & ? & ? & ? & ? & ? & ?\\
  F T F & ? & ? & ? & ? & ? & ?\\
  F F T & ? & ? & ? & ? & ? & ?\\
  F F F & ? & ? & ? & ? & ? & ?\\
  \hline
\end{tabular}
\end{center}
  \end{enumerate}

(Solution): %Don't forget to state if there are any equivalent propositions

\item (5 + 5 points)
In the future, all robots are either knights who always tell the truth or knaves who always lie. You are a time traveler and you arrive in the future. You come across two robots named %Alice and Bob.
Leela and Fry. Leela says: 'If Fry is a knave, then I am a knave.' What are Leela and Fry?

  Ensure that you fully explain \textbf{every} step of your reasoning. You are allowed to write the steps in English. Make sure to consider all cases. 
  You can get 5 bonus points for translating this problem into Boolean algebra and then create a truth table (please note that you cannot exceed 100 points for this homework).
  
(Solution)
%Your Solution goes here



 

\item (20 points)
  An explorer lands on a strange island with two towns.
  All inhabitants of Truth always tell the truth, and all inhabitants of Lies always lie.
  The explorer asks the first islander, Alice, which town she comes from, and she gives an answer that the explorer cannot understand.
  The second islander, Bob, says, ``Alice said she is from Lies.''
  The third islander, Charlie, says to Bob, ``You are a liar!''
  From which town is Charlie?

  Ensure that you fully explain every step of your reasoning.

(Solution)
%Your Solution goes here



\item (20 points)
  Determine whether the following compound propositions are satisfiable. If it is satisfiable, provide an assignment; otherwise, show why it is not satisfiable.
  \begin{enumerate}
  \item $(p \lor q \lor r) \land (\lnot p \lor q \lor s) \land (p \lor \lnot q \lor \lnot s) \land (\lnot q \lor \lnot r \lor s)$


  
  \item $(\lnot p \lor \lnot q \lor \lnot s) \land (p \lor q \lor r \lor s) \land (p \lor q \lor s) \land (\lnot p \lor q \lor r)$
  

  \end{enumerate}

\item (25 points)
  Translate the following arguments into logical statements using predicates, logical connectives, and quantifiers as appropriate.
  %use the math mode starting and ending with $$
  %Check the supporting file of LaTeX commands on Canvas (or search online)
  %$\forall%
  %$\exists$
  %$\land$ for AND
  %$\lor$ for OR
  %$\neg$ for NOT or negation
  

  \begin{enumerate}
  \item
    All cats understand French.\\
    Some chickens are cats.\\
    Some chickens understand French.

  \item
    All wise men walk on their feet.\\
    All unwise men walk on their hands.\\
    No man walks on both.

    
  \end{enumerate}

\end{enumerate}
\end{document}


