%%%%%%% In this homework, all the .tex files are combined in 1 file. %%%%%%% Please add your solutions at the indicated areas. 
%%%%%%% 
%%%%%%% 

\documentclass[letterpaper,12pt]{article}

\usepackage{alltt}
\usepackage{amsfonts}
\usepackage{amsmath}
\allowdisplaybreaks[1]%0-4 4 is most permissive of breaking equations
\usepackage{amssymb}
\usepackage{amsthm}
\usepackage{array}
\usepackage{calc}
\usepackage{color}
\usepackage{graphicx}
\usepackage{stmaryrd}
\usepackage{supertabular}
\usepackage{url}%Wanted by the bibilography
\usepackage[all]{xy}

%######################################
%Quote environment
\newsavebox{\quauth}
\newenvironment{Quotation}[1]
	{\sbox{\quauth}{\textit{#1}}\begin{quote}}
	{\hspace*{\fill}\nolinebreak[1]\hspace*{\fill}\usebox{\quauth}\end{quote}}
%######################################

%######################################
%Theorem setup, needs package amsthm
%Theorems
\theoremstyle{plain}
\newtheorem{thm}{Theorem}[section]
\newtheorem{algo}[thm]{Algorithm}
\newtheorem{cor}[thm]{Corollary}
\newtheorem{lem}[thm]{Lemma}
\newtheorem{prop}[thm]{Proposition}


%Definitions
\theoremstyle{definition}
\newtheorem{appl}[thm]{Application}
\newtheorem{conj}[thm]{Conjecture}
\newtheorem{defn}[thm]{Definition}
\newtheorem{exmp}[thm]{Example}
\newtheorem{exer}[thm]{Exercise}

%Remarks
\theoremstyle{remark}
\newtheorem{rem}[thm]{Remark}
\newtheorem{note}[thm]{Note}
\newtheorem*{case}{Case}
\newtheorem*{claim}{Claim}


%For referencing Theorems/Defs/etc...
\providecommand{\thmref}[1]{ (Thm. \ref{#1})}
\providecommand{\defref}[1]{ (Def. \ref{#1})}
\providecommand{\exref}[1]{Example \ref{#1}}
%######################################

%######################################
%Define a few column types for convenience
%Needs package array

%These make the columns all be in math mode
%The extra %stopzone are to make vim do syntax
%highlighting correctly with this code, otherwise
%it chokes
\newcolumntype{C}{>{$}c<{$}}
%stopzone%stopzone%stopzone
\newcolumntype{L}{>{$}l<{$}}
%stopzone%stopzone%stopzone
\newcolumntype{R}{>{$}r<{$}}
%stopzone%stopzone%stopzone
%######################################

%######################################
%Various shortcuts
%Some may need package amssymb

%Shortcuts for various sets
\providecommand{\A}{\ensuremath{\mathfrak{A}}}%ideal A
\providecommand{\B}{\ensuremath{\mathcal{B}}}%basis
\providecommand{\C}{\ensuremath{\mathbb{C}}}%complex
\providecommand{\F}{\ensuremath{\mathbb{F}}}%finite field
\providecommand{\N}{\ensuremath{\mathbb{N}}}%natural
\providecommand{\RI}{\ensuremath{\mathcal{O}}}%ring of integers
\providecommand{\p}{\ensuremath{\mathfrak{P}}}%prime ideal
\providecommand{\Q}{\ensuremath{\mathbb{Q}}}%rationals
\providecommand{\R}{\ensuremath{\mathbb{R}}}%reals
\providecommand{\T}{\ensuremath{\mathcal{T}}}%topology
\providecommand{\Z}{\ensuremath{\mathbb{Z}}}%integers
\providecommand{\Zpos}{\ensuremath{\mathbb{Z}^{+}}}%positive integers


%Miscellaneous shortcuts
\providecommand{\lcm}{\ensuremath{\text{lcm}}}%least common multiple
\providecommand{\st}{\ensuremath{\backepsilon}}
\providecommand{\ud}{\ensuremath{\,\,\mathrm{d}}}%For derivatives
%######################################

%######################################
% Various operators
% some may need package amsmath

%Algebra
\providecommand{\amod}{\ensuremath{\diagup}}%Algebraic mod
\providecommand{\embed}{\ensuremath{\hookrightarrow}}%inclusion map
\providecommand{\gen}[1]{\ensuremath{\left<#1\right>}}%Group generated by #1
\providecommand{\idx}[2]{\ensuremath{\left[#1:#2\right]}}%index of #2 in #1
\providecommand{\im}{\ensuremath{\mathrm{im}}}%Image
\providecommand{\iso}{\ensuremath{\simeq}}%isomorphism
\providecommand{\normal}{\ensuremath{\vartriangleleft}}%Normal subgroup
\providecommand{\subgp}{\ensuremath{\le}}%Sub group
\providecommand{\vect}[1]{\ensuremath{\left\langle#1\right\rangle}}%Vector

%Logic
\providecommand{\land}{\ensuremath{\wedge}}
\DeclareMathSymbol\lnot{\mathbin}{symbols}{"3A}%p528
\providecommand{\lor}{\ensuremath{\vee}}
\providecommand{\lxor}{\ensuremath{\oplus}}

%Number Theory
\providecommand{\jacobi}[2]{\ensuremath{\left(\frac{#1}{#2}\right)}}
\providecommand{\legendre}[2]{\ensuremath{\left(\frac{#1}{#2}\right)}}

%Set Theory
\providecommand{\intersect}{\ensuremath{\bigcap}}
\providecommand{\less}{\ensuremath{\diagdown}}
\providecommand{\union}{\ensuremath{\bigcup}}

%Miscellaneous
\providecommand{\abs}[1]{\ensuremath{\left\lvert#1\right\rvert}}
\providecommand{\card}[1]{\ensuremath{\left\lvert#1\right\rvert}}
\providecommand{\ceiling}[1]{\ensuremath{\left\lceil#1\right\rceil}}
\providecommand{\define}{\ensuremath{\stackrel{\text{\tiny def}}{=}}}
\providecommand{\floor}[1]{\ensuremath{\left\lfloor#1\right\rfloor}}
\providecommand{\norm}[1]{\ensuremath{\left\lVert#1\right\rVert}}%Analysis norm
\providecommand{\order}[1]{\ensuremath{\left\lvert#1\right\rvert}}

%Asymptotic
\providecommand{\Oh}{\ensuremath{\mathcal{O}}}
\providecommand{\oh}{\ensuremath{o}}
%######################################
%######################################
%Page layout
% If using hyperef, load it before this block
\usepackage[paper=letterpaper,tmargin=1in,bmargin=42pt,lmargin=1in,rmargin=1in,headheight=30pt,headsep=30pt,footskip=20pt]{geometry}
% Note 1in = 72pt, therefore bmargin=42pt is 1in - 30pt
\usepackage{ifpdf}
\ifpdf
  \geometry{driver=pdftex}
\else
  \geometry{driver=dvips}
\fi

\usepackage{fancyhdr}
\lhead{COMP 2300: Discrete Structures}
\chead{Homework 2}
\rhead{Autumn 2023}
\lfoot{}
\cfoot{}
\rfoot{\thepage}
\renewcommand{\headrulewidth}{0pt}
\renewcommand{\footrulewidth}{0pt}
%######################################

\usepackage{tabularx}


%%%%%%%%%%%%%%%%%%%%%%%%%%%%%%%%%%%%%%%%%%%%%%%%%%%%%%%%%%%%%%%%%%%%%%%%%%%%%%%%%
%%%%%%%%%%%%%%%%%%%%%%%%%%%%%%%%%%%%%%%%%%%%%%%%%%%%%%%%%%%%%%%%%%%%%%%%%%%%%%%%%
%%%%%%%%%%%%%%%%%%%%%%%%%%%%%%%%%%%%%%%%%%%%%%%%%%%%%%%%%%%%%%%%%%%%%%%%%%%%%%%%%
%%%%%%%%%%%%%%%%%%%%%%%%%%%HOMEWORK DOCUMENT START%%%%%%%%%%%%%%%%%%%%%%%%%%%%%%%

\begin{document}
\pagestyle{fancy}
\begin{enumerate}




\item (30 points)
  Translate the following arguments into logical statements using predicates, logical connectives, and quantifiers as appropriate.

  \begin{enumerate}
      \item
    All robots say ``beep''.\\
    Some computers are robots.\\
    Some computers say ``beep''.
%%%%%%%%%%%%%%%%% Add your solution below%%%%%%%%%%%%%%%%%%


  \item
    All wise men walk on their feet.\\
    All unwise men walk on their hands.\\
    No man walks on both.
  \end{enumerate}
%%%%%%%%%%%%%%%%% Add your solution below%%%%%%%%%%%%%%%%%%




\item (10 points) Simplify the expression $(p \lor \lnot q) \land (p \lor q)$
%Solution:







\item (10 points)
  Express each of the following using mathematical and logical operators, predicates, and quantifiers.
  Define the domain of each variable used.
  \begin{enumerate}
  \item
    All positive real numbers are the square of a negative number.
  \item
    There are no negative numbers that are the square of any positive number.
  \end{enumerate}

  

%Solution:




  
  
\item (10 points) 
Prove (show) that the following two statements are equivalent. Use the properties of logical equivalences (i.e. do not use a truth table). 
\begin{center}
    $(p \to q) \land (p \to r) \equiv p\to (q \land r)$\\
\end{center}
%Solution:




\item (10 points) 
  \begin{enumerate}
  \item Rewrite the following expressions into equivalent expressions where all negation operators immediately precede predicates. Show all intermediate steps.

    \[\lnot \exists x \forall y \bigl( P(x) \land Q(y) \bigr)\]
    
%Solution:



 \item Express the negation of this statement so that all negation symbols immediately precede predicates. For example, $P(x) \land \lnot Q(x)$ is an acceptable form, $\lnot(P(x) \land Q(x))$ is not acceptable.
       \[\forall x \exists y \bigl(P(x,y) \to Q(x,y) \bigr)\] 
     \end{enumerate}
%Solution:





\item (15 points)
  In the following argument, at each step identify in the ``Reason'' column the rule of inference or other rule of logic used to arrive at each step.
  See Examples 6, 7, 12, and 13 in \S 1.6 in the book for the correct form of the answer.

  The premises are" \\
  ``All Mallards are Ducks,'' \\
  ``No Ducks which are not Mallards have green heads,'' \\
  and \\
  ``Daffy is a duck and has a green head.''\\
  The conclusion is ``Daffy is a Mallard.''

  Define predicates $D(x)$ to be True iff ($\iff$, biconditional) $x$ is a Duck, $M(x)$ to be True iff $x$ is a Mallard, $G(x)$ to be true iff $x$ has a green head.
  The domain of $x$ is all birds.
  Define $d$ to be ``Daffy,'' in the domain of $x$.
  We then have:

  \begin{tabular}{cl}
    & $\forall x \bigl( M(x) \to D(x) \bigr)$\\
    & $\forall x \Bigl( \bigl( D(x) \land \lnot M(x) \bigr) \to \lnot G(X) \Bigr)$\\
    & $D(d) \land G(d)$\\
    \hline
    $\therefore$ & $M(d)$
  \end{tabular}

  \renewcommand{\arraystretch}{2}
  \begin{tabularx}{\textwidth - \leftmargin}{clX}
    & Argument Steps: & Reason/Rule/Law of Logic:\\
    a) & $\forall x \Bigl(\bigl(D(x) \land \lnot M(x)\bigr) \to \lnot G(x) \Bigr)$ & %%%Solution goes here
    \\
    \cline{3-3}
    b) & $\bigl(D(d) \land \lnot M(d)\bigr) \to \lnot G(d)$ & %%%Solution goes here
    \\
    \cline{3-3}
    c) & $D(d) \land G(d)$ & %%%Solution goes here
    \\
    \cline{3-3}
    d) & $G(d)$ & %%%Solution goes here
    \\
    \cline{3-3}
    e) & $\lnot \bigl(D(d) \land \lnot M(d)\bigr)$ & %%%Solution goes here
    \\
    \cline{3-3}
    f) & $\lnot D(d) \lor M(d)$ & %%%Solution goes here
    \\
    \cline{3-3}
    g) & $D(d)$ & %%%Solution goes here
    \\
    \cline{3-3}
    h) & $M(d)$ & %%%Solution goes here
    \\
    \cline{3-3}
  \end{tabularx}
  \renewcommand{\arraystretch}{1}




\item (15 points)
  In the previous question, what if we instead had an argument with premises ``All Mallards are Ducks,'' ``No Ducks which are not mallards have green heads,'' and ``Huey is a duck and does not have a green head.''
  The conclusion is ``Huey is not a Mallard.''
  Using the predicates defined in the previous questions, give the argument form, and then either give the argument steps showing that the argument is valid, or demonstrate that the argument is not valid.\\
  
  %Solution:\\
\newpage

\begin{exmp}
Use rules of inference to show that if $\forall x(P(x) \to (Q(x) \land S(x)))$ and $\forall x(P(x) \land R(x))$ are true, then
$\forall x(R(x) \land S(x))$ is true.
\end{exmp}

\begin{tabular}{cl}
    & $\forall x(P(x) \to (Q(x) \land S(x)))$\\
    & $\forall x(P(x) \land R(x))$\\
    \hline
    \therefore & $\forall x(R(x) \land S(x))$
  \end{tabular}


  \renewcommand{\arraystretch}{2}
  
  \begin{tabularx}{\textwidth - \leftmargin}{clX}
    & Argument Steps: & Reason/Rule/Law of Logic:\\

    a) & $\forall x(P(x) \land R(x))$ & \text{Premise}
    \\
    \cline{3-3}
    b) & \text{for arbitrary }$c, P(c) \land R(c)$  & \text{Universal Instantiation of a)}
    \\
    \cline{3-3}
    
    c) & \text{for arbitrary }$c, R(c)$ & \text{simplification of b)}
    \\
    \cline{3-3}
    d) & $\forall x(P(x) \to (Q(x) \land S(x)))$ & \text{premise}
    \\
    \cline{3-3}
    e) & for arbitrary $c$,$(P(c) \to (Q(c) \land S(c)))$ & \text{Universal Instantiation of d)}
    \\
    \cline{3-3}
    
    f) & \text{for arbitrary }$c, P(c)$ & \text{simplification of b)}
    \\
    \cline{3-3}
    g) & for arbitrary $c$, $(Q(c) \land S(c))$ & \text{modus ponens of e) and f)}
    \\
    \cline{3-3}
    h) & for arbitrary $c$, $S(c)$ & \text{simplification of g)}
    \\
    \cline{3-3}
    i) & for arbitrary $c$, $R(c)\land S(c)$ & \text{By conjunction of c) and h)}
    \\
    \cline{3-3}
    j) & $\forall x R(x)\land S(x)$ & \text{Universal Generalization of i)}
    \\
    \cline{3-3}
  \end{tabularx}
  \renewcommand{\arraystretch}{1}

\newline

"No Ducks which are not mallards have green heads,"
\end{enumerate}

\newpage
\begin{exmp}
\begin{align}
a\lor(b\land c)\equiv( a\lor b) \land (a \lor c) \text{ distributive law}
\end{align}

\begin{align}
(p\land q)\lor(p\land \lnot q)&\equiv (p \land (q \lor \lnot q))\text{ distributive (backwards!)}\\
&\equiv (p \land (T))\text{ Negation Laws}\\
&\equiv p\text{ Identity}
\end{align}

\end{exmp}

\newpage

WTS:\\
$p\to q$, $q\to r$ $\vdash r$ 

This is false because, $p=False$, $q=False$, $r=False$,
is an assignment which satisfies the premises and not the conclusion.

$\forall x(M(x)\to R(x)) \vdash \forall R(x)$ 

This is false, let \mathfrak{U}:=\{a\} and 
$R(a)=False$ and $M(a)=False$.
Then all premises are satisfied, but the conclusion is not.
$M(a)\to R(a)$ is true, by implication (truth table)\\
$\forall x(M(x)\to R(x)$



\newpage

look for any rule that\\

\begin{align}
p \to T\equiv& T \lor \lnot p \text{   rewriting of implication }\\
 \equiv& T \text{   domination}
\end{align}

\end{document}
