%%%%%%% DO NOT REMOVE COMMANDS OR PACKAGE FROM THE DOCUMENT

\documentclass[letterpaper,10pt]{article}

\usepackage{alltt}
\usepackage{listings}
\usepackage{amsfonts}
\usepackage{amsmath}
\allowdisplaybreaks[1]%0-4 4 is most permissive of breaking equations
\usepackage{amssymb}
\usepackage{amsthm}
\usepackage{array}
\usepackage{calc}
\usepackage{color}
\usepackage{graphicx}
\usepackage{stmaryrd}
\usepackage{supertabular}
\usepackage{url}%Wanted by the bibilography
\usepackage[all]{xy}

%%Packages for tables
\usepackage{wrapfig}
\usepackage{multirow}
\usepackage{tabu}

%Note: Throughout the preamble there are occassional references to pages out of
%books. Unless otherwise noted they are from The LaTeX Companion 2nd ed by
%Mittelbach and Goossens

%######################################
%Font Selection.
%Concrete
%\usepackage{beton}
%\usepackage{euler}
%Computer Modern Bright
%\usepackage[T1]{fontenc}
%\usepackage{cmbright}
%######################################

%######################################
%Quote environment
\newsavebox{\quauth}
\newenvironment{Quotation}[1]
	{\sbox{\quauth}{\textit{#1}}\begin{quote}}
	{\hspace*{\fill}\nolinebreak[1]\hspace*{\fill}\usebox{\quauth}\end{quote}}
%######################################

%######################################
%Theorem setup, needs package amsthm
%Theorems
\theoremstyle{plain}
\newtheorem{thm}{Theorem}[section]
\newtheorem{algo}[thm]{Algorithm}
\newtheorem{cor}[thm]{Corollary}
\newtheorem{lem}[thm]{Lemma}
\newtheorem{prop}[thm]{Proposition}


%Definitions
\theoremstyle{definition}
\newtheorem{appl}[thm]{Application}
\newtheorem{conj}[thm]{Conjecture}
\newtheorem{defn}[thm]{Definition}
\newtheorem{exmp}[thm]{Example}
\newtheorem{exer}[thm]{Exercise}

%Remarks
\theoremstyle{remark}
\newtheorem{rem}[thm]{Remark}
\newtheorem{note}[thm]{Note}
\newtheorem*{case}{Case}
\newtheorem*{claim}{Claim}


%For referencing Theorems/Defs/etc...
\providecommand{\thmref}[1]{ (Thm. \ref{#1})}
\providecommand{\defref}[1]{ (Def. \ref{#1})}
\providecommand{\exref}[1]{Example \ref{#1}}
%######################################

%######################################
%Define a few column types for convenience
%Needs package array

%These make the columns all be in math mode
%The extra %stopzone are to make vim do syntax
%highlighting correctly with this code, otherwise
%it chokes
\newcolumntype{C}{>{$}c<{$}}
%stopzone%stopzone%stopzone
\newcolumntype{L}{>{$}l<{$}}
%stopzone%stopzone%stopzone
\newcolumntype{R}{>{$}r<{$}}
%stopzone%stopzone%stopzone
%######################################

%######################################
%Various shortcuts
%Some may need package amssymb

%Shortcuts for various sets
\providecommand{\A}{\ensuremath{\mathfrak{A}}}%ideal A
\providecommand{\B}{\ensuremath{\mathcal{B}}}%basis
\providecommand{\C}{\ensuremath{\mathbb{C}}}%complex
\providecommand{\F}{\ensuremath{\mathbb{F}}}%finite field
\providecommand{\N}{\ensuremath{\mathbb{N}}}%natural
\providecommand{\RI}{\ensuremath{\mathcal{O}}}%ring of integers
\providecommand{\p}{\ensuremath{\mathfrak{P}}}%prime ideal
\providecommand{\Q}{\ensuremath{\mathbb{Q}}}%rationals
\providecommand{\R}{\ensuremath{\mathbb{R}}}%reals
\providecommand{\T}{\ensuremath{\mathcal{T}}}%topology
\providecommand{\Z}{\ensuremath{\mathbb{Z}}}%integers
\providecommand{\Zpos}{\ensuremath{\mathbb{Z}^{+}}}%positive integers


%Miscellaneous shortcuts
\providecommand{\lcm}{\ensuremath{\text{lcm}}}%least common multiple
\providecommand{\st}{\ensuremath{\backepsilon}}
\providecommand{\ud}{\ensuremath{\,\,\mathrm{d}}}%For derivatives
%######################################

%######################################
% Various operators
% some may need package amsmath

%Algebra
\providecommand{\amod}{\ensuremath{\diagup}}%Algebraic mod
\providecommand{\embed}{\ensuremath{\hookrightarrow}}%inclusion map
\providecommand{\gen}[1]{\ensuremath{\left<#1\right>}}%Group generated by #1
\providecommand{\idx}[2]{\ensuremath{\left[#1:#2\right]}}%index of #2 in #1
\providecommand{\im}{\ensuremath{\mathrm{im}}}%Image
\providecommand{\iso}{\ensuremath{\simeq}}%isomorphism
\providecommand{\normal}{\ensuremath{\vartriangleleft}}%Normal subgroup
\providecommand{\subgp}{\ensuremath{\le}}%Sub group
\providecommand{\vect}[1]{\ensuremath{\left\langle#1\right\rangle}}%Vector

%Logic
\providecommand{\land}{\ensuremath{\wedge}}
\DeclareMathSymbol\lnot{\mathbin}{symbols}{"3A}%p528
\providecommand{\lor}{\ensuremath{\vee}}
\providecommand{\lxor}{\ensuremath{\oplus}}
\providecommand{\todo}[1]{\textcolor{magenta}{\textbf{#1}}}

%Number Theory
\providecommand{\jacobi}[2]{\ensuremath{\left(\frac{#1}{#2}\right)}}
\providecommand{\legendre}[2]{\ensuremath{\left(\frac{#1}{#2}\right)}}

%Set Theory
\providecommand{\intersect}{\ensuremath{\bigcap}}
\providecommand{\less}{\ensuremath{\diagdown}}
\providecommand{\union}{\ensuremath{\bigcup}}

%Miscellaneous
\providecommand{\abs}[1]{\ensuremath{\left\lvert#1\right\rvert}}
\providecommand{\card}[1]{\ensuremath{\left\lvert#1\right\rvert}}
\providecommand{\ceiling}[1]{\ensuremath{\left\lceil#1\right\rceil}}
\providecommand{\define}{\ensuremath{\stackrel{\text{\tiny def}}{=}}}
\providecommand{\floor}[1]{\ensuremath{\left\lfloor#1\right\rfloor}}
\providecommand{\norm}[1]{\ensuremath{\left\lVert#1\right\rVert}}%Analysis norm
\providecommand{\order}[1]{\ensuremath{\left\lvert#1\right\rvert}}

%Asymptotic
\providecommand{\Oh}{\ensuremath{\mathcal{O}}}
\providecommand{\oh}{\ensuremath{o}}
%######################################
%######################################
%Page layout
% If using hyperef, load it before this block
\usepackage[paper=letterpaper,tmargin=1in,bmargin=42pt,lmargin=.5in,rmargin=.5in,headheight=30pt,headsep=30pt,footskip=20pt]{geometry}
% Note 1in = 72pt, therefore bmargin=42pt is 1in - 30pt
\usepackage{ifpdf}
\ifpdf
  \geometry{driver=pdftex}
\else
  \geometry{driver=dvips}
\fi

%%%%%%%%%%%%
%%%%%%%%%%%%
%%%%%%%%%%%%
%%%%%%%%%%%% Here you enter the running title
%
%%%%%%%%%%%%
\usepackage{fancyhdr}
\lhead{COMP 2300}
%It is recommended to add your name
%%%%%%%%%%%%
\chead{Notes}
\rhead{Fall 2023}
\lfoot{}
\cfoot{}
\rfoot{\thepage}
\renewcommand{\headrulewidth}{0pt}
\renewcommand{\footrulewidth}{0pt}
%######################################
%######################################
%######################################
%######################################
%######################################

\begin{document}
\pagestyle{fancy}
%######################################
%######################################
%######################################
%######################################
%HOMEWORK START

\section{The Foundations}
\subsection{Logic and Proofs}

\begin{defn}[proposition]
A \textbf{proposition} is a declarative sentence (that is, a sentence that declares a fact) that is either true
or false, but not both.
\end{defn}

\begin{defn}[propositional variables]
We use letters to denote \textbf{propositional variables} (or statement variables), that is, variables that represent propositions, just as letters are used to denote numerical variables.
\end{defn}
\begin{defn}[truth value] The \textbf{truth value} of a proposition is true, denoted by $T$, if it is a true proposition, and the truth value of a proposition
is false, denoted by $F$, if it is a false proposition.
\end{defn}

\begin{defn}[negation]
Let $p$ be a proposition. The \textbf{negation} of $p$, denoted by $ \lnot p$, is the statement 
\newline\todo{"It is not the case that p"}. 
\newline The proposition $\lnot p$ is read “not p.” The truth value of the negation of $p$, $\lnot p$, is the opposite of the truth value of $p$.
\end{defn}

\begin{exmp}[truth table of $\lnot$]
Below is an example of a truth table. 

Typically, the leftmost columns correspond to the propositional variables that we use in our model. 

Each row in the table corresponds to a different assignment of the propositional variables, each possible combination of T and F values.

For example, the first row below corresponds to the world (or model) in which $p$ is True. The second row corresponds to world (or model) in which $p$ is False. 

The columns to the right of the propositional variables tell us the value of $\lnot p$ corresponding to the truth value of $p$ in this row.

\begin{center}
\begin{tabular}{ | c | c | c |} 
  \hline
 $p$& $ \lnot p$ \\
 \hline
  T & F \\ 
  F & T \\ 
  \hline
\end{tabular}
\end{center}
\end{exmp}

Suppose that $p$ is the proposition "candy is free"


Then the first row corresponds to a world in which the proposition "candy is free" is True, in which case the negation of this proposition "it is not the case that candy is free" is False.

The second row corresponds to a world in which the proposition "candy is free" is \todo{?False}, in which case the negation of this proposition "it is not the case that candy is free" is \todo{True}

\begin{defn}[conjunction,($\land$)]
Let $p$ and $q$ be propositions. The \textbf{conjunction} of $p$ and $q$, denoted by $p \land q$, is the proposition
“$p$ and $q$.” The conjunction $p \land q$ is True when both $p$ and $q$ are True and is False otherwise.
\end{defn}

\begin{exmp}[truth table of $\land$]
Below is the \textbf{truth table} of $p \land q$. In the leftmost columns are the propositional values $p$ and $q$. Each row represents a different world (or model or assignment) in which $p$ and $q$ take on the values True and False. 

\begin{center}
\begin{tabular}{ | c | c | c |} 
  \hline
 $p$ $q$ & $ p \land q$ \\
 \hline
  T T & T \\ 
  T F & F \\ 
  F T & F \\ 
  F F & F \\ 
  \hline
\end{tabular}
\end{center}

The first row represent a world in which both $p$ and $q$ are true. 

The second row represents a world in which $p$ is True and $q$ is \todo{F}.

The third row represents a world in which $p$ is \todo{F} and $q$ is \todo{T}.

The last row represents \todo{a world in which p and q are both false}

Suppose that $p$ represents the proposition "candy is free" and that $q$ represents the proposition "we all get along".
Then the first row represents a world in which $p$ and $q$ are true, in which case the proposition "candy is free and we all get along " is true. The second row, however, represents a world in which the proposition "candy is free" is \todo{True} and "we all get along is" \todo{False} in which case the proposition "candy is free and we all get along " is \todo{False}.
\end{exmp}


\newpage
\begin{defn}[disjunction $\lor$]
Let $p$ and $q$ be propositions. The \textbf{disjunction} of $p$ and $q$, denoted by $p \lor q$, is the proposition
“p or q.” The disjunction $p \lor q$ is false when both p and q are false and is true otherwise.
\end{defn}

\begin{exmp}[truth table of or $\lor$]
Below is the truth table of $p \lor q$. In the leftmost columns are the propositional values $p$ and $q$. Each row represents a different world (or model or assignment) in which $p$ and $q$ take on the values True and False. 

\begin{center}
\begin{tabular}{ | c | c | c |} 
  \hline
 $p$ $q$ & $ p \lor q$ \\
 \hline
  T T & \todo{T} \\ 
  T F & \todo{T} \\ 
  F T & \todo{T} \\ 
  F F & \todo{F} \\ 
  \hline
\end{tabular}
\end{center}

Suppose that $p$ represents the proposition "I am going to win my soccer match" and that $q$ represents the proposition "I am going to cry".

The first row represent a world in which both $p$ and $q$ are true.
This means that I will \todo{win} my soccer match and then I will \todo{cry}. The proposition "I am going to win my soccer match or I am going to cry" is \todo{True}.

The second row represents a world in which $p$ is True and $q$ is False. This means that I will \todo{Win} my soccer match and then I will \todo{Not Cry}. The proposition "I am going to win my soccer match or I am going to cry" is \todo{True}.

The third row represents a world in which $p$ is \todo{False} and $q$ is \todo{True}. This means that I will \todo{Lose} my soccer match and then I will \todo{Cry}. The proposition "I am going to win my soccer match or I am going to cry" is \todo{True}.

The last row represents a world in which $p$ is \todo{False} and $q$ is \todo{False}. This means that I will \todo{Lose} my soccer match and then I will \todo{Not Cry}. The proposition "I am going to win my soccer match or I am going to cry" is \todo{False}.
\end{exmp}
\begin{defn}[exclusive or $\lxor$]
Let $p$ and $q$ be propositions. The \textbf{exclusive or} of $p$ and $q$, denoted by $p \lxor q$, is the proposition
that is true when exactly one of $p$ and $q$ is true and is false otherwise.
\end{defn}
\begin{exmp}[truth table of $\lxor$]

\begin{center}
\begin{tabular}{ | c | c | c |} 
  \hline
 $p$ $q$ & $ p \lxor q$ \\
 \hline
  T T & \todo{F} \\ 
  T F & \todo{T} \\ 
  F T & \todo{T} \\ 
  F F & \todo{F} \\ 
  \hline
\end{tabular}
\end{center}
\end{exmp}
If $p$ is the statement "My side is soup" and $q$ is the statement "My side is salad", then $p \lxor q$ represents the proposition "My side is soup or salad (but not both)", which is true in models where exactly one of the propositions "my side is soup" and "my side is salad" is true.

\begin{defn}[conditional $p \to q$]
Let $p$ and $q$ be propositions. The conditional statement $p \to q$ is the proposition “if $p$, then
$q$.” The conditional statement $p \to q$ is false when $p$ is true and $q$ is false, and true otherwise.
In the conditional statement $p \to q$, $p$ is called the hypothesis (or antecedent or premise)
and $q$ is called the conclusion (or consequence).
\end{defn}
\begin{exmp}[truth table of $\to$]
\begin{center}
\begin{tabular}{ | c | c | c |} 
  \hline
 $p$ $q$ & $ p \to q$ \\
 \hline
  T T & \todo{?\_\_\_\_} \\ 
  T F & \todo{?\_\_\_\_} \\ 
  F T & \todo{?\_\_\_\_} \\ 
  F F & \todo{?\_\_\_\_} \\ 
  \hline
\end{tabular}
\end{center}
If $p$ is the proposition "You give me 1000 dollars" and $q$ is the statement "Your roof will stop leaking." Then $p \to q$ represents the proposition, 
\newline 
\todo{"If you give me 1000 dollars your roof  will stop leaking"}.
\newline
The proposition $p\to q$ is False when the proposition "You give me 1000 dollars" is \todo{True} and "Your roof will stop leaking." is \todo{False}.
\end{exmp}

\newpage

\begin{exmp}[conditional $p\to q$ ] \ \\

Consider the two propositions and their respective propositional variables

$ p:=\text{"You eat 20 hot dogs for dinner"} $

$ q:=\text{"You don't need a little snacky poo"} $

The proposition (or statement)
\newline
$p\to q:= $\todo{If you eat 20 hot dogs for dinner then you don't need a little snacky poo }
\newline
In a world in which you only ate 19 hotdogs for dinner and you don't want a snack, $p\to q$ is \todo{True}.

In a world in which you eat 20 hotdogs and you do still need a little snack, $p \to q$ is \todo{False}.

\end{exmp}

\begin{defn}[biconditional $p\iff q$]\ \\
Let $p$ and $q$ be propositions. The \textbf{biconditional} statement $p \iff q$ is the proposition “$p$ if
and only if $q$.” The biconditional statement $p \iff q$ is true when $p$ and $q$ have the same truth
values, and is false otherwise. Biconditional statements are also called bi-implications.
\end{defn}

\begin{exmp}[truth table of $\iff$]
\begin{center}
\begin{tabular}{ | c | c | c |} 
  \hline
 $p$ $q$ & $ p \iff q$ \\
 \hline
  T T & \todo{T} \\ 
  T F & \todo{F} \\ 
  F T & \todo{F} \\ 
  F F & \todo{T} \\ 
  \hline
\end{tabular}
\end{center}
If $p$ is the proposition "Harry potter dies." and $q$ is the statement "Voldemort dies." Then $p \iff q$ represents the proposition, \todo{"Harry potter dies if and only if Voldemort dies"}.

The proposition $p\iff q$ is False when the proposition "Harry potter dies." is True and "Voldemort dies." is \todo{False} OR when the proposition "Harry potter dies." is False and "Voldemort dies." is \todo{True}.

\end{exmp}

\newpage
\subsection{Applications of Propositional Logic}

\subsubsection{Logic Puzzles}
\begin{exmp}[logic puzzle]

On an island in the future there are two groups of people. One group always lies and the other group always tells the truth.

You come across 2 people.
One person says "I don't lie, but she does!"
The other person says "I don't like, but she does!"

What are possible scenarios of who is the liar and who tells the truth?

Can you write the scenario above as a compound proposition?

\begin{proof} \ \\
I will interpret "I don't lie, but she does!" as equivalent to the statement "I don't lie \textbf{and} she does!"

$A:=\text{person 1 tells the truth}$

$B:=\text{person 2 tells the truth}$

Then person one says, $A\land (\lnot B)$
and person two says $B\land (\lnot A)$

Supposing that $A$ is true (person 1 is telling the truth), $A\land (\lnot B)$ must be true, in which case $\lnot B $ is true. This means person 1 is telling the truth and person 2 is lying. 
We now need to confirm that person 2 lied. If person 2 lied, $B$ is false and $B \land \lnot(A)$ must also be false. Therefore, $\lnot(B \land \lnot(A))$ must be true. Using De Morgans law, $\lnot B \lor \lnot A$ must be True, and since $\lnot B$ we see that this is the case. Therefore it is possible that person 1 told the truth and person 2 lied.

Supposing that $A$ is false (person 1 is lying), $A\land (\lnot B)$ must be false, which is satisfied by $A$ being false. Therefore $B$ may take the value false and the statement $A\land (\lnot B)$ is false (because person lied in the first half of the sentence and the best interpretation of 'but' in this context is 'and' but it makes me want to cry. Not all sentences translate perfectly.)

\end{proof}

\end{exmp}

\subsection{Propositional Equivalences}

\begin{defn}[tautology and contradiction]

A compound proposition that is always true, no matter what the truth values of the propositional
variables that occur in it, is called a \textbf{tautology}.A compound proposition that is always
false is called a \textbf{contradiction}. A compound proposition that is neither a tautology nor a
contradiction is called a contingency.

\end{defn}
\begin{defn}[equivalence]
The compound propositions $p$ and $q$ are called logically equivalent if $p \iff q$ is a tautology.
The notation $p \equiv q$ denotes that p and q are logically equivalent. This means that the two statements evaluate identically in every possible world!
\end{defn}
\begin{exmp}[equivalences]

Construct truth tables for the following compound propositions.
  \begin{enumerate}
  \item $\lnot \lnot p $
  \item $p \to \lnot p$
  \item $\lnot (p \lor q)$
  \item $ \lnot p \land \lnot q$
  \end{enumerate}
  
  Are any of the propositions equivalent? 
  Are any Tautologies or Contradictions?
  If so, which ones and why?
  
  

\begin{figure}[htb]
    \begin{minipage}{1.5in}
        \begin{center} 
        \begin{tabular}{ | c | c | c |} 
  \hline
 $p$ $q$ & $ \lnot p$ & $\lnot \lnot p$  \\
 \hline
  T T &F  &T \\ 
  T F & F & T\\ 
  F T & T & F \\ 
  F F &  T& F\\ 
  \hline
\end{tabular}
        \end{center}
    \end{minipage}
    \begin{minipage}{1.5in}
        \begin{center} 
        \begin{tabular}{ | c | c | c |} 
  \hline
 $p$ $q$ & $ \lnot p$ & $p \to \lnot p$  \\
 \hline
  T T & F & F\\ 
  T F & F & F\\ 
  F T & T & T\\ 
  F F & T & T\\ 
  \hline
\end{tabular}
        \end{center}
    \end{minipage}
    \begin{minipage}{1.7in}
        \begin{center}
        \begin{tabular}{ | c | c | c |} 
  \hline
 $p$ $q$ & $p \lor q$ & $\lnot (p \lor q)$ \\
 \hline
  T T &T &F\\ 
  T F & T&F\\ 
  F T & T&F \\ 
  F F &F &T \\ 
  \hline
\end{tabular}
        \end{center}
    \end{minipage}
    \begin{minipage}{1.7in}
        \begin{center}
        \begin{tabular}{ | c | c | c |c |} 
  \hline
 $p$ $q$ & $\lnot p$ & $\lnot q$ & $ \lnot p \land \lnot q$ \\
 \hline
  T T & F& F& F \\ 
  T F &F & T&F \\ 
  F T & T& F& F\\ 
  F F &T & T& T\\ 
  \hline
\end{tabular}
        \end{center}
    \end{minipage}
    
\end{figure} 

\end{exmp}

\newpage
\begin{exmp}[satisfiability]

\begin{enumerate}
  Determine whether the following compound propositions are satisfiable. If it is satisfiable, provide an assignment, otherwise show why it is not satisfiable. (Recall that a compound expression is satisfiable if there exists a model or world in which the expression is True.)
  
  \item $(p \land \lnot q \land \lnot r \land s) \lor (\lnot p \land q \land \lnot r \land s) \lor (p \land \lnot q \land \lnot r)$
  
 Hint: Do you see a trick?
  \begin{proof}
  
 Recall $A\lor B \lor C$ evaluates to true when at least one of the collection $A$, $B$, or $C$ evaluates to true. Therefore if $(p \lor q \lor \lnot s) $ is true, the larger expression above is true. 
 we see this expression is satisfied when $p$ is True, $q$ is false, $r$ is true and $s$ is true.
 Similarly, we can find when the second expression evaluates to true. $(\lnot p \land q \land \lnot r \land s)$ is true when $p$ is true, $q$ is true, $r$ is false and $s$ is false.
 The third expression, $(p \land \lnot q \land \lnot r)$, is true when $p$ is true, $q$ is false and $r$ is false. Notice this has two cases. One where $s$ is false and the other when $s$ is true. 
  \begin{center}
        \begin{tabular}{ | c | c | c |c |} 
  \hline
 $p$ $q$ $r$ $s$ \\
 \hline
  T F T T\\ 
  T T F F \\ 
  T T F T \\ 
  T T F F \\ 
  \hline
\end{tabular}
        \end{center}
 
 \end{proof}
 
 
  \item $(p \lor q \lor \lnot s) \land (p \lor \lnot q \lor r \lor s) \land (p \lor q \lor s)$

 Hint: (you can always try one assignment at a time and see if the expression evaluates to true!)

 
 \vspace{3cm}
\end{enumerate}
\end{exmp}
\begin{exmp}[NEW MATERIAL, covered in detail Monday]
  Translate the following arguments into logical statements using predicates, logical connectives, and quantifiers as appropriate.

  \begin{enumerate}
  \item
    All frogs like water.\\
    Some puppets are frogs.\\
    No puppets like water.
    
    Hint: let 
    $F(x):=\text{x is a frog}$
    
    $W(x):=\text{x likes water}$
    
    $P(x):=\text{x is a puppet}$
    \vspace{3cm}

  \item
    Some odd numbers are prime.\\
    not all prime numbers are odd.\\
    not all odd numbers are primte.\\
    \vspace{3cm}

    
  \end{enumerate}

\end{exmp}



\end{document}